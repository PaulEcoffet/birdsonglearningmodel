%!TeX spellcheck = en-US
\documentclass[a4paper]{article}
\usepackage[dvipsnames]{xcolor}
\usepackage[american]{babel}
\usepackage[utf8]{inputenc}
\usepackage[T1]{fontenc}
\usepackage[dvipsnames]{xcolor}
\usepackage{lmodern}
\usepackage{amssymb,amsmath}
\usepackage{comment} % enables the use of multi-line comments (\ifx \fi)
\usepackage{fullpage} % changes the margin
\usepackage{todonotes}
\usepackage{import}
\usepackage{multicol}

\usepackage[backend=bibtex,bibstyle=authoryear,url=false, isbn=false]{biblatex}
\addbibresource{SingingBirds.bib}
%\DeclareLanguageMapping{american}{american-apa}

\usepackage{xifthen}
\usepackage{soul}
\sethlcolor{Apricot}
\newcommand\bla[1]{\ifthenelse{\isempty{#1}}{\hl{**~bla~bla~**}}{\hl{**~#1~**}}}
\usepackage[unicode=true]{hyperref}
\usepackage[all]{hypcap} % ref link to the top of the figure

\usepackage{csquotes} % Dependency for APA


\hypersetup{breaklinks=true,
            pdfauthor={Paul Ecoffet},
            pdftitle={Preregistration report},
            colorlinks=true,
            citecolor=blue,
            urlcolor=blue,
            linkcolor=magenta,
            pdfborder={0 0 0}
            }
\urlstyle{same} % don't use monospace font for urls

%\AtEveryBibitem{\clearfield{title}}
\AtEveryBibitem{\clearfield{journaltitle}}
\AtEveryBibitem{\clearfield{volume}}
\AtEveryBibitem{\clearfield{pages}}
\AtEveryBibitem{\clearfield{number}}
\renewbibmacro{in:}{}

\begin{document}
\noindent
\large\textbf{Présoutenance} \hfill \textbf{Paul Ecoffet} \\
\normalsize Cogmaster M2 \hfill Supervisors: Stéphane Doncieux, Benoît Girard \\
Tuteur: Mehdi Khamassi \hfill Institut de systèmes intelligents et robotique, UPMC\\
Session: June \hfill Language: English\\
Suggested reviewers: Arthur~Leblois \& Jean-Pierre~Nadal \hfill 2017-01-20\\
\href{https://osf.io/ja8k9/}{OSF project's page}
\begin{multicols}{2}
\section*{Problem Statement}

The Zebra Finches are songbirds which learn the song of their tutor. They learn
it from 25 days post hatch (DPH) to 90 DPH \parencite{liu_juvenile_2004}. Zebra
finches are commonly used as a model of speech acquisition.

\textcite{deregnaucourt_how_2005} showed that sleep plays an important role in
the learning of tutor songs. Indeed, they showed that sleeping has a negative
impact on song restitution by zebra finches in the short term but a positive
impact on the long run. Song restitution is less complex and less similar to the
tutor song from one morning to the previous day evening, but the greater this
loss in performance was overall for one bird, the better this bird was able to
reproduce the tutor song at the end of its learning.


\textcite{dave_song_2000} have found neurons in the motor cortex which fires
sequences during sleep that correspond to their activity pattern when the birds
sing in adult zebra finches. This shows that motor neurons that are highly
correlated with bird's own song (BOS) are activated during the night. These
identified replays suggest that some learning may occur during sleep that use
past experiences.

Our hypothesis is that during its sleep, the zebra finch restructures the
knowledge it has acquired so far thanks to replay mechanisms. We hypothesize
that this restructuring can account for the loss of performance in the short
term and an improvement of performance in the long term.

The goal of this internship is to offer a model of the zebra finch song
learning which can explain different behavioral data observed such as the
correlation between the loss of performance every night and the overall
performance at the end of learning.

\section*{Investigation/Research}

Our goal is to build a biological plausible model. We will use a bird song
synthesizer made by \textcite{boari_automatic_2015}. This synthesizer is a
biophysical model of zebra finch vocal apparatus. It can be parameterized with
relatively few values to produce realistic bird songs. As it models the zebra
finch vocal apparatus, it is likely that the parameters we send to this
synthesizer are similar to the instructions sent by the zebra finch motor cortex
to vocal apparatus muscles. Zebra finches song have already been reproduced
using this synthesizer and a look up table \parencite{boari_automatic_2015}. The
synthetic songs they produced activated neurons in HVC which are highly
selective to bird's own song (BOS). This shows that the synthesized songs are
accurate reproductions of BOS.

The authors of the synthesizer have found that what can be seen as syllable in
the sensory space can be seen as one or several ``gestures'' in this parameters
space \parencite{amador_low_2014, boari_automatic_2015}. Gestures are continuous
and monotonous variations in the parameter space. These gestures can represent
the real motor representation of the song. They define the notion of gesture
trajectory extrema (GTE), which are the period in which the bird switch from a
gesture to another. We hypothesize that gesture and therefore GTE identification
may play an important role in song learning, as they signal changes in the
progression of the parameters through time.

Our hypothese is that during the day, the bird is optimizing its sound
reproduction toward subparts of the song. During the night, it uses the
knowledge acquired during the day to restructure its song decomposition which
will determine its new goals. The song decomposition will be the one for which
the bird knows the gestures that yield the closest sound for each part. It
suggests the presence of replays of already known gestures during sleep.

The restructuration of the song decomposition during the night will have a short
term negative impact as the bird will have to optimize new goals but will
choose more and more adaptive segmentation that lead to an overall better
performance.

\section*{Proposed Solution}

Our goal is to design a simple optimization algorithm that fits one specific
gesture and a gesture identification algorithm. The gesture identification
algorithm will try to segment the tutor song in efficient gestures based on the
bird current knowledge acquired by the optimization algorithm. This two-step
algorithm is similar in some points to an Expectation-Maximization algorithm
\parencite{dempster_maximum_1977}.

This part will only cover the learning of gestures, but not the transitions
between these gestures. We have yet to find how to learn the pattern of the
syllables and the song. This new algorithm should be able to use the knowledge
built by the gesture learning system. The algorithm that learns the syllable
transition should also be able to reproduce the different learning strategies
that a zebra finch can have: a serial strategy, where it only learns one
specific syllable at a time, or a motif strategy, where it tries to reproduce
the whole tutor song at every try.


\section*{Expected Implementation}

We have already implemented the Python binder to the compiled synthesizer by
\textcite{boari_automatic_2015}. These sound-waves will then be processed by an
auditory system. We plan to use Mel-Frequency Cepstrum Coefficients (MFCC),
which are used in speech recognition for Humans. MFCC have already been used to
classify birdsongs \parencite{chou_studies_2008}.

We plan to use a Nearest-Neighbor algorithm to hill climb toward the goals the
algorithm defines and remember each try it makes for the segmentation algorithm.
The segmentation algorithm will try to do cuts in the tutor song so that each
cut is the closest possible to a sound the bird can already make according to
its previous tries.

\section*{Analysis \& Testing}

To assess the quality of our model, we have selected two criteria to meet.
First, we want our algorithm to reproduce the results from
\textcite{deregnaucourt_how_2005}. This includes observing the increase of song
similarity \parencite{tchernichovski_procedure_2000} over the development of the
bird, that the song similarity decreases overnight, but that this decrease is
positively correlated with the song restitution at the end of the learning. To
do that, we will use the same statistical tests that
\textcite{deregnaucourt_how_2005} have done on the same set of features: Amongst
other tests, we will measure the signed song similarity between the last 100
songs produced and the first 100 songs produced the next day, and see if the
difference is negative (going against overall developpement). We will then do a
correlation of pearson test between the overall magnitude of post-sleep
deterioration and the eventual similarity to the model song, we expect to see a
positive correlation between the two.

Then, we want that our algorithm identifies similar GTE distribution as the
identified ones by the automatic GTE extractor made by
\textcite{boari_automatic_2015}. We expect the mean amount of GTE per syllable
inferred by our algorithm and Boari's algorithm to be equal.

\section*{Final Evaluation}

Once the model is fully working and respond to our expectations in reproducing
the literature, we will study its behaviors to design new hypotheses that can be
tested on real zebra finches. For instance, we can make postulates about the
replay neural pattern in the night during its development. We can also make the
hypothesis that the bird use the same song decomposition through out the day.

\section*{Contributions}

S. Derégnaucourt provided us behavioral data for model comparison and will help
us reproducing his analyses.\\
S. Boari and his colleagues built the song synthesizer and a GTE detection
algorithm.\\
I will code the model and the analysis with the supervision of Stéphane Doncieux
and Benoît Girard.\\

\printbibliography{}

\end{multicols}

\end{document}
