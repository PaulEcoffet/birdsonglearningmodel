%!TeX spellcheck = en-US
\documentclass[a4paper]{article}
\usepackage[dvipsnames]{xcolor}
\usepackage[american]{babel}
\usepackage[utf8]{inputenc}
\usepackage[T1]{fontenc}
\usepackage[dvipsnames]{xcolor}
\usepackage{lmodern}
\usepackage{amssymb,amsmath}
\usepackage{comment} % enables the use of multi-line comments (\ifx \fi)
\usepackage{fullpage} % changes the margin
\usepackage{todonotes}
\usepackage{import}

\usepackage[backend=biber,bibstyle=authoryear,url=false, isbn=false]{biblatex}
\addbibresource{SingingBirds.bib}
\DeclareLanguageMapping{american}{american-apa}

\usepackage{xifthen}
\usepackage{soul}
\sethlcolor{Apricot}
\newcommand\bla[1]{\ifthenelse{\isempty{#1}}{\hl{**~bla~bla~**}}{\hl{**~#1~**}}}
\usepackage[unicode=true]{hyperref}
\usepackage[all]{hypcap} % ref link to the top of the figure

\usepackage{csquotes} % Dependency for APA


\hypersetup{breaklinks=true,
            pdfauthor={Paul Ecoffet},
            pdftitle={Report},
            colorlinks=true,
            citecolor=blue,
            urlcolor=blue,
            linkcolor=magenta,
            pdfborder={0 0 0}}
\urlstyle{same} % don't use monospace font for urls


\begin{document}
\noindent
\large\textbf{Présoutenance} \hfill \textbf{Paul Ecoffet} \\
\normalsize Cogmaster M2 \hfill Supervisors: Stéphane Doncieux, Benoît Girard \\
Tuteur: Mehdi Khamassi \hfill Institut de systèmes intelligents et robotique, UPMC\\
Session: June \hfill Language: English\\
Suggested reviewers: … \& … \hfill 2017-01-20\\
\href{https://osf.io/ja8k9/}{OSF project's page}


\section*{Problem Statement}

The Zebra Finches are songbirds which learn the song of their tutor. They learn
it from 25 days post hatch (DPH) to 90 DPH \parencite{liu_juvenile_2004}. Zebra
finches are commonly used as a model of speech acquisition.

\textcite{deregnaucourt_how_2005} showed that sleep plays an important role in
the learning of tutor songs. Indeed, they showed that sleeping has a negative
impact on song restitution by zebra finches in the short term but a positive
impact on the long run. Indeed, song restitution is less complex and less
similar to the tutor song from one morning to the previous day evening, but the
greater this loss in performance was overall for one bird, the better this bird
was able to reproduce the tutor song at the end of its learning.


\textcite{dave_song_2000} have found replay sequences of neurones in the motor
cortex which correspond to their activity pattern when the birds sing in adult
zebra finches during their sleep. This shows that neurones that are highly
correlated with bird's own song (BOS) are activated during the night. These
findings suggest that some learning may be at play during sleep.

Our hypothesis is that during its sleep, the zebra finch restructures the
knowledge it has acquired so far with replay. We hypothesize that this
restructuring can account for the loss of performance in the short term and an
improvement of performance in the long term.

The goal of this internship is to offer a model of the zebra finch song
learning which can explain different behavioral data observed such as the
correlation between the loss of performance every night and the overall
performance at the end of learning, as well as the different phases of bird song
learning (babbling, protosyllables, crystallized song).

\section*{Investigation/Research}

Our goal is to build a biological plausible model. We will use a bird song
synthesizer made by \textcite{boari_automatic_2015}. This synthesizer is a
biophysical model of zebra finch vocal apparatus. It can be parametrized with
relatively few values to produce realistic bird songs. As it models the zebra
finch vocal apparatus, it is likely that the parameters we send to this
synthesizer are similar to the instructions sent by the zebra finch motor cortex
to the vocal apparatus muscles. Zebra finches song have already been reproduced
using this synthesizer and a look up table \parencite{boari_automatic_2015}. The
synthetic songs they produced activated neurons in HVC which are highly
selective to bird's own song (BOS). This shows that the synthesized songs are
accurate reproductions of BOS.

The authors of the synthesizer have found that what can be seen as syllable in
the sensory space can be seen as one or several ``gestures'' in this parameters
space \parencite{amador_low_2014, boari_automatic_2015}. These gestures can
represent the real motor representation of the song. We hypothesize that GTE
identification may play an important role in song learning, as they signal
changes in the progression of the parameters through time.

\section*{Proposed Solution}

Our goal is to design a simple optimization algorithm that fits one specific
gesture and a gesture identification algorithm. The gesture identification
algorithm will try to segment the tutor song in efficient gestures based on the
bird current knowledge acquired by the optimization algorithm. This two step
algorithm is similar in some points to an Expectation-Maximization algorithm
\parencite{dempster_maximum_1977}.

Our idea is that the maximization of the identified gestures occurs during the
day and explains the overall advancement in performance. The segmentation
algorithm will have a short term negative impact but will choose more and more
adaptive segmentation that lead to an overall better performance.

This part will only cover the learning of gesture, but not the link between
these gestures. We have yet to find how to learn the pattern of the syllables
and the song. This algorithm should be able to use in a smart way the knowledge
built by the gesture learning system. The algorithm that learn the syllable
transition should also be able to reproduce the different learning strategies
that a zebra finch can have: a serial strategy, where it only learn one specific
syllable at a time, or have a motif strategy, where it learn to reproduce the
whole tutor song at every try.


\section*{Expected Implementation}

The learning algorithm that we want to suggest must be biologically plausible,
therefore we will use the Song Synthesizer from \textcite{boari_automatic_2015}
to generate real sound-waves. We have already implemented the Python binder to
the compiled synthesizer. These sound-waves will then be processed by an
auditory system. We plan to use Mel-Frequency Cepstrum Coefficients
(MFCC), which are used in speech recognition for Humans.
MFCC have already been used to classify birdsongs \parencite{chou_studies_2008}.

We plan to use a Nearest-Neighbor algorithm to hill climb toward the goals the
algorithm defines and remember each try it makes for the segmentation algorithm.
The segmentation algorithm will try to do cuts in the tutor song so that each
cut is the closest to a sound the bird can already make.

\section*{Analysis \& Testing}

To assess the quality of our model, we have selected several criteria to meet.
First, we want our algorithm to reproduce the results from
\textcite{deregnaucourt_how_2005}. This include observing the increase of song
similarity \parencite{tchernichovski_procedure_2000} over the development of the
bird, that the song similarity decrease over night, but that this decrease is
positively correlated with the song restitution at the end of the learning. To
do that, we will use the same statistical tests that
\textcite{deregnaucourt_how_2005} have done on the same set of features.

Then, we expect our model to yield at first protosyllables, then syllables and
go into the crystallized song. This can be measured by more and more consistent
syllable production, with less variance for a syllable production for several
features (duration, pitch, frequency modulation, Wiener entropy, goodness of
pitch and continuity).\todo{less variance seems good but i have not found
the same idea in other paper}

Finally, we want that our algorithm identifies similar GTE distribution as the
identified ones by the automatic GTE extractor made by
\textcite{boari_automatic_2015}. We expect the mean amount of GTE per syllable
inferred by our algorithm and Boari's algorithm to be equal.

\section*{Final Evaluation}

The goal of building this model is being able to build new hypotheses that can
be tested in behavioral or neurobiological experiment on the zebra finch. Once
the model is fully working and respond to our expectations in reproducing the
literature, we will study its behaviors so as to design new hypotheses that can
be tested on real zebra finches.

\section*{Contributions}
S. Derégnaucourt provides us behavioral data for model comparison\\
S. Boari and his colleagues built the song synthesizer and a GTE detection
algorithm.\\
I will code the model and the analysis with the supervision of Stéphane Doncieux
and Benoît Girard.\\

\clearpage%
\printbibliography{}

\end{document}
