\chapter{Introduction}\label{introduction}

\section{Zebra Finch song learning}\label{zebra-finch-song-learning}

\subsection{Characteristic of zebra finch song
learning}\label{characteristic-of-zebra-finch-song-learning}

\begin{itemize}
\tightlist
\item
  Learn one song for its whole life, Close-end learner (opposition
  open-end learer)
\item
  Learn the song of father
\item
  Song divided in motifs, syllables and notes
\item
  subsong (babbling), plastic song, cristallysation
\item
  Sensory phase (memorisation of tutor song), sensory motor phase
\item
  If no auditory feedback, cannot learn.
\end{itemize}

\subsection{Why is zebra finch song learning
studied}\label{why-is-zebra-finch-song-learning-studied}

\begin{itemize}
\tightlist
\item
  Model of Human speech learning

  \begin{itemize}
  \tightlist
  \item
    Actual song learning, not innate
  \item
    Song with complex structures
  \end{itemize}
\item
  Well studied Neuroanatomy
\item
  Easy to study experimentally

  \begin{itemize}
  \tightlist
  \item
    Easily domesticated
  \item
    Learn one song
  \item
    Learn quickly (90DPH)
  \item
    Easy to track song development
  \end{itemize}
\end{itemize}

\section{Neurobiology of the Zebra
Finch}\label{neurobiology-of-the-zebra-finch}

\subsection{Neuroanatomy of the Zebra Finch song
system}\label{neuroanatomy-of-the-zebra-finch-song-system}

\begin{itemize}
\tightlist
\item
  Connection between RA, HVC, Area X, \ldots{} Inhibition, excitation
\end{itemize}

\subsection{Pattern of activation in RA and
HVC}\label{pattern-of-activation-in-ra-and-hvc}

\begin{itemize}
\tightlist
\item
  HVC clock like, temporal structure (Ali et al.)
\item
  RA activation while singing at very precise time and sparse coding

  \begin{itemize}
  \tightlist
  \item
    Motor control (Ali et al.) Ali et al. shows real two different
    learning: spectral and temporal
  \end{itemize}
\end{itemize}

\section{Models of song learning}\label{models-of-song-learning}

Only very few models have been created. Even less are actual
computational models.

\subsection{Reinforcement learning}\label{reinforcement-learning}

\begin{itemize}
\tightlist
\item
  Proposed but no real explanation of what could be the state space, the
  action space, the reward function (Dave\&Margoliash).
\item
  Used in paradigm to test different hypothesis (averse reward to force
  change in behaviour of the bird)
\end{itemize}

\subsection{Song preferences in selection
(Marler)}\label{song-preferences-in-selection-marler}

\begin{itemize}
\tightlist
\item
  Behavioural model to explain how the bird select its template
\item
  TODO: Add more
\end{itemize}

\subsection{Coen's model}\label{coens-model}

\begin{itemize}
\tightlist
\item
  Clustering technique with babbling (multimodal)

  \begin{itemize}
  \tightlist
  \item
    Cluster the tutor song syllables thanks to their characteristics
  \item
    Babbling, create a mapping between the motor space and the
    identified cluster
  \end{itemize}
\item
  Use of a real synthesizer but not actually built to model zf vocal
  apparatus
\item
  No quantitative means to see how good is the song reproduction
\item
  The learning is only babbling, nothing is driving the model in a
  specific direction.
\end{itemize}

\section{Song synthesizer}\label{song-synthesizer}

\subsection{Description of Perl song synthesizer to reproduce Zebra
Finch
song}\label{description-of-perl-song-synthesizer-to-reproduce-zebra-finch-song}

\begin{itemize}
\tightlist
\item
  Presentation of anatomy simulation, mass and spring\ldots{}
\item
  Parameters

  \begin{itemize}
  \tightlist
  \item
    Air sac pressure
  \item
    Syringeal Labial Tension
  \end{itemize}
\item
  Parameters are close to actual motor actions, so close to actual motor
  command
\end{itemize}

\subsection{Zebra Finches are sensible to song produced by the
synthesizer}\label{zebra-finches-are-sensible-to-song-produced-by-the-synthesizer}

\begin{itemize}
\tightlist
\item
  Show results of Amador where RA neurons were activated by Synth song
  but not by conspecific song.
\end{itemize}

\subsection{Gestures and song
structure}\label{gestures-and-song-structure}

\begin{itemize}
\tightlist
\item
  Boari's Gesture concept and automatic extraction of the gestures
\item
  Could have been correlated to HVC activation but in fact no.
\end{itemize}

\section{Influence of Sleep in the Zebra Finch song
development}\label{influence-of-sleep-in-the-zebra-finch-song-development}

\subsection{Margoliash results with song
replay}\label{margoliash-results-with-song-replay}

\begin{itemize}
\tightlist
\item
  RA neurons activated while singing
\item
  Also activated when the bird is asleep and listen to his own song
\item
  Spontaneous activity with part of the song: Replays
\item
  Replays can be consolidation of memory
\item
  Replays can have another role
\end{itemize}

\subsection{Derégnaucourt results about positive impact of sleep for
development}\label{deruxe9gnaucourt-results-about-positive-impact-of-sleep-for-development}

\begin{itemize}
\tightlist
\item
  Extraction of syllables characteristics and track over time
\item
  Global trend for the trajectory of a syllable over time
\item
  Each day, the syllables characteristics are closer
\end{itemize}

\section{A computational model of birdsong learning to explain the sleep
influence}\label{a-computational-model-of-birdsong-learning-to-explain-the-sleep-influence}

\subsection{Interest of a computational model of birdsong
learning}\label{interest-of-a-computational-model-of-birdsong-learning}

\begin{itemize}
\tightlist
\item
  Computational model helps understanding what are the
  \emph{implementation constraints} of the learning mechanisms

  \begin{itemize}
  \tightlist
  \item
    Use of synthesizer
  \item
    Realistic computational budget
  \end{itemize}
\item
  Easily make hypotheses that can be tested experimentally afterwards
\item
  Abstracted and controlled environment
\end{itemize}

\subsection{Goal: Build a modular two-step learning model and look for
learning algorithm that can account for Derégnaucourt's
results.}\label{goal-build-a-modular-two-step-learning-model-and-look-for-learning-algorithm-that-can-account-for-deruxe9gnaucourts-results.}

\chapter{Our Model}\label{our-model}

\section{Global Architecture}\label{global-architecture}

\subsection{Usage of Boari's implementation of the birdsong
synthesizer}\label{usage-of-boaris-implementation-of-the-birdsong-synthesizer}

\subsection{Measurement of song quality with standard
measures}\label{measurement-of-song-quality-with-standard-measures}

\begin{itemize}
\tightlist
\item
  Entropy, Pitch, Goodness, Amplitude, Frequency Modulation, Amplitude
  Modulation
\item
  Imported from Matlab implementation, with qualitatively similar
  results
\end{itemize}

\subsection{Two-step learning model}\label{two-step-learning-model}

\begin{itemize}
\tightlist
\item
  Bird has several song models it trains to reach tutor
\item
  tutor song is known
\item
  day algorithm for parameters optimisation
\item
  night algorithm for structure optimisation
\item
  Hypothesis:~structure optimisation yield unlearning short term, better
  learning long term
\end{itemize}

\section{Song Model}\label{song-model}

\subsection{Song Model}\label{song-model-1}

\subsection{Gesture paradigm inherited from
synthesizer}\label{gesture-paradigm-inherited-from-synthesizer}

\subsection{Song structure}\label{song-structure}

\begin{itemize}
\tightlist
\item
  List of gestures and their duration
\item
  Fixed duration of the song because of measurement
\end{itemize}

\subsection{Gesture composed of two generators for the motor
commands}\label{gesture-composed-of-two-generators-for-the-motor-commands}

\begin{itemize}
\tightlist
\item
  Abstracted in sum of sin \& linear func
\end{itemize}

\section{Day learning algorithm}\label{day-learning-algorithm}

\subsection{goal}\label{goal}

\begin{itemize}
\tightlist
\item
  Optimise gestures parameters
\end{itemize}

\subsection{Hillclimbing}\label{hillclimbing}

\begin{itemize}
\tightlist
\item
  really simple
\item
  Choose song model, choose gesture
\item
  Choose close parameters, if better keep, if worse trash
\item
  Knows if better by comparison of weighted standard measurements
\item
  Not whole song but only gesture trained to make faster computations

  \begin{itemize}
  \tightlist
  \item
    Actually creates unlearning
  \end{itemize}
\end{itemize}

\subsection{Prediction}\label{prediction}

\begin{itemize}
\tightlist
\item
  Should improve song production but get stuck in local maximum because
  bad structure
\end{itemize}

\section{Night learning algorithms}\label{night-learning-algorithms}

\subsection{Goal}\label{goal-1}

\begin{itemize}
\tightlist
\item
  Find better structure to describe song motor command
\end{itemize}

\subsection{Several variations of algorithm have been
tested}\label{several-variations-of-algorithm-have-been-tested}

\begin{itemize}
\tightlist
\item
  Evolutionary algorithm

  \begin{itemize}
  \tightlist
  \item
    Simple solution for structure variation
  \end{itemize}
\item
  with or without diversity
\end{itemize}

\subsection{Algorithm}\label{algorithm}

\begin{itemize}
\tightlist
\item
  Evolutionary algorithm Microbial GA
\item
  Increase population size and add variation in structure

  \begin{itemize}
  \tightlist
  \item
    Remove, add, change, copy gesture
  \item
    Song always the same length for comparison reasons.
  \end{itemize}
\item
  Compare by tournament

  \begin{itemize}
  \tightlist
  \item
    The winner put a variation of itself in place of the loser
  \item
    Compare number of neighbour * score, lower the better
  \end{itemize}
\end{itemize}

\subsection{Predictions}\label{predictions}

\begin{itemize}
\tightlist
\item
  Structure variation yields unlearning short term but positive impact
  long term
\item
  Diversity will increase this
\end{itemize}

\section{Parameters}\label{parameters}

\begin{itemize}
\tightlist
\item
  Tried to be realistic
\item
  most are fit through gridsearch
\item
  Realistics: Number of days, number of syllables sung during all dev
\item
  Gridsearch optimisation
\item
  Default value for gesture parameters
\item
  Learning rate

  \begin{itemize}
  \tightlist
  \item
    Prevent part of unlearing
  \item
    Could be fitted to match real song learning rate
  \item
    Coefficient for score optimisation
  \end{itemize}
\item
  Algorithm way better in score than Boari but qualitatively very
  different to the ear
\item
  Look at which parameters boari's method was better than algo and put
  priority on them
\item
  Amplitude and entropy
\item
  Diversity threshold to maximise variance in diversity score

  \begin{itemize}
  \tightlist
  \item
    Value: 5000
  \item
    Other parameters
  \end{itemize}
\item
  Number of song models during day and night: Depend of runs
\item
  Boundaries for parameters values: Fixed
\item
  Number of tournaments during night: depend of runs

  \begin{itemize}
  \tightlist
  \item
    Correlated with replay? By how much?
  \end{itemize}
\end{itemize}

\chapter{Analyses and results}\label{analyses-and-results}

\section{Learning method is as good Boari's method or
better}\label{learning-method-is-as-good-boaris-method-or-better}

\begin{itemize}
\tightlist
\item
  Using standard measure criteria in the birdsong community
\item
  Simple description of motor params sufficient to produce good songs
\item
  Qualitatively same amount of gestures

  \begin{itemize}
  \tightlist
  \item
    Can be due to luck
  \end{itemize}
\end{itemize}

\section{Too little training per model cause
divergence}\label{too-little-training-per-model-cause-divergence}

\begin{itemize}
\tightlist
\item
  maybe due to global vs local error
\end{itemize}

\section{Derégnaucourt results not
reproduced}\label{deruxe9gnaucourt-results-not-reproduced}

\begin{itemize}
\tightlist
\item
  Syllables extracted by time of begin and end
\item
  Without or with diversity
\item
  No night deterioration
\item
  Night deterioration has no impact in overall learning
\end{itemize}

\chapter{Discussion}\label{discussion}

\section{The synthesizer which cannot produce every
sounds}\label{the-synthesizer-which-cannot-produce-every-sounds}

\begin{itemize}
\tightlist
\item
  Our score really close to boari's method (not way better or way
  worst), maybe we reached synthesizer limits
\end{itemize}

\section{The parameters description we
choose}\label{the-parameters-description-we-choose}

\begin{itemize}
\tightlist
\item
  more simple/complexe possible than sum of sin and affine?
\end{itemize}

\section{The unlearning during day due to the gesture
learning}\label{the-unlearning-during-day-due-to-the-gesture-learning}

\section{Fixed duration of songs in
learning}\label{fixed-duration-of-songs-in-learning}

\begin{itemize}
\tightlist
\item
  Dynamic Time Warping can correct that
\end{itemize}

\section{Big artificial separation between structuration and gestures
optimisation}\label{big-artificial-separation-between-structuration-and-gestures-optimisation}

\section{Diversity not strong enough? What if only diversity during
night?}\label{diversity-not-strong-enough-what-if-only-diversity-during-night}

\begin{itemize}
\tightlist
\item
  Maybe not convergence
\item
  Maybe what we are looking for
\end{itemize}

\chapter{Conclusion}\label{conclusion}

\section{Learning algorithm with two step
learning}\label{learning-algorithm-with-two-step-learning}

\begin{itemize}
\tightlist
\item
  Very few of them
\item
  Working with realistic synthesizer
\item
  modular architecture, easy to test new models
\end{itemize}

\section{Restructuration didn't yield the expected
effect}\label{restructuration-didnt-yield-the-expected-effect}

\begin{itemize}
\tightlist
\item
  More parameters search might be able to fix it
\end{itemize}
