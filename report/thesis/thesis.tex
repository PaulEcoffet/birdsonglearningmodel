\documentclass{report}
\usepackage[dvipsnames]{xcolor}
\usepackage[american]{babel}
\usepackage[utf8]{inputenc}
\usepackage[T1]{fontenc}
\usepackage[dvipsnames]{xcolor}
\usepackage{lmodern}
\usepackage{amssymb,amsmath}
\usepackage{comment} % enables the use of multi-line comments (\ifx \fi)
\usepackage{fullpage} % changes the margin
\usepackage{todonotes}
\usepackage{import}
\usepackage{multicol}
\usepackage{enumitem}


\usepackage[backend=biber,style=apa]{biblatex}
\addbibresource{SingingBirds.bib}
\DeclareLanguageMapping{american}{american-apa}

\usepackage{xifthen}
\usepackage{soul}
\sethlcolor{Apricot}
\newcommand\bla[1]{\ifthenelse{\isempty{#1}}{\hl{**~bla~bla~**}}{\hl{**~#1~**}}}
\usepackage[unicode=true]{hyperref}
\usepackage[all]{hypcap} % ref link to the top of the figure

\usepackage{csquotes} % Dependency for APA

\usepackage{titlesec}

\titleformat{\chapter}{\normalfont\huge}{\thechapter.}{20pt}{\Huge}


\hypersetup{breaklinks=true,
            pdfauthor={Paul Ecoffet},
            pdftitle={Master thesis},
            colorlinks=true,
            citecolor=blue,
            urlcolor=blue,
            linkcolor=magenta,
            pdfborder={0 0 0}
            }
\urlstyle{same} % don't use monospace font for urls


\title{Master Thesis - Zebra finches}
\author{Paul Ecoffet}
\date{The 6th of June, 2017}




\begin{document}
\maketitle

\begin{abstract}
The Zebra Finches are songbirds which learn the song of their tutor. They learn
it from 25 days post hatch (DPH) to 90 DPH \parencite{liu_juvenile_2004}. Zebra
finches are commonly used as a model of speech acquisition.

\textcite{deregnaucourt_how_2005} showed that sleep plays an important role in
the learning of tutor songs. Indeed, they showed that sleeping has a negative
impact on song restitution by zebra finches in the short term but a positive
impact on the long run. Song restitution is less complex and less similar to the
tutor song from one morning to the previous day evening, but the greater this
loss in performance was overall for one bird, the better this bird was able to
reproduce the tutor song at the end of its learning.


In addition to that, \textcite{dave_song_2000} have found neurons in the motor
cortex which fires sequences during sleep that correspond to their activity
pattern when the birds sing in adult zebra finches. This shows that motor
neurons that are highly correlated with bird's own song (BOS) are activated
during the night. These identified replays suggest that some learning may occur
during sleep that use past experiences.

Our hypothesis is that during its sleep, the zebra finch restructures the
knowledge it has acquired so far thanks to replay mechanisms. We hypothesize
that this restructuring can account for the loss of performance in the short
term and an improvement of performance in the long term.

The goal of this internship is to offer a model of the zebra finch song
learning which can explain different behavioral data observed such as the
correlation between the loss of performance every night and the overall
performance at the end of learning.
\end{abstract}

\tableofcontents

\chapter{Introduction}

\section{Zebra finch song learning}
The Zebra Finches are songbirds which learn the song of their tutor. They learn
it from 25 days post hatch (DPH) to 90 DPH \parencite{liu_juvenile_2004}. Zebra
finches are commonly used as a model of speech acquisition.

Why model of speech acquisition:
* fast learning (90 days)
* Learning during development \cite{margoliash_offline_2003}
* Very reproductible result with a tutor song that the bird will mimic its whole
  life \cite{margoliash_sleep_2010}
    * Close ended learners vs open-ended learner \cite{margoliash_sleep_2010}
* Many-to-many relationship between auditory feedback and muscle outputs \cite{margoliash_offline_2003}

* Can be easily raised domesticaly (Rosenfield)
* More localised brain structure (Rosenfield)
    * Can study circuitries



The stages of learning
* subsong, plastic, cristallization \cite{margoliash_sleep_2010}


\section{Neuroscience}

\section{Sleep}
\textcite{deregnaucourt_how_2005} showed that sleep plays an important role in
the learning of tutor songs. Indeed, they showed that sleeping has a negative
impact on song restitution by zebra finches in the short term but a positive
impact on the long run. Song restitution is less complex and less similar to the
tutor song from one morning to the previous day evening, but the greater this
loss in performance was overall for one bird, the better this bird was able to
reproduce the tutor song at the end of its learning.

\textcite{dave_song_2000} have found neurons in the motor cortex which fires
sequences during sleep that correspond to their activity pattern when the birds
sing in adult zebra finches. This shows that motor neurons that are highly
correlated with bird's own song (BOS) are activated during the night. These
identified replays suggest that some learning may occur during sleep that use
past experiences.

* Replays can explain learning \cite{margoliash_offline_2003}
* Consolidate learning
* Needed to reach new sounds \cite{margoliash_offline_2003}
* replays with weird burst inside \cite{margoliash_evaluating_2002}

\section{Sum up}
Our hypothesis is that during its sleep, the zebra finch restructures the
knowledge it has acquired so far thanks to replay mechanisms. We hypothesize
that this restructuring can account for the loss of performance in the short
term and an improvement of performance in the long term.

The goal of this internship is to offer a model of the zebra finch song
learning which can explain different behavioral data observed such as the
correlation between the loss of performance every night and the overall
performance at the end of learning.

\section{Previous models for birdsong learning with Zebra Finches}

* Preference for same species songs \cite{margoliash_evaluating_2002,
marler_three_1997}
* \cite{marler_three_1997}
* \cite{coen_learning_2007}


\chapter{Our proposed model}

I build the model in Python.

\section{Global architecture}

\subsection{Terminology}

\begin{description}
  \item[Tutor song] The tutor song is the goal song the agent try to reproduce.
  \item[Song Model] A song model is the representation of the motor commands an
  agent has to produce a song. A song model is composed of several gestures,
  with the associated parameters and duration.
  \item[Song structure] The song structure is the agencement of the gestures.
  Without modifying the parameters of a gesture, the song structure can change
  by inserting a new gesture, deleting one or changing the length of a gesture.
  \item[Gesture] ?
  \item[Learning model] ?
\end{description}

\subsection{Overview}

\subsection{Memorisation of the tutor song template}

* Tutor song is known by the bird ``by heart'' \cite{deregnaucourt_how_2005, margoliash_sleep_2010}
``Exposure to a song model, either from a live tutor (e.g. father or other
male) or song playback, typically results in the production of a good imitation
of the tutor song. For learning to occur, birds only need to hear the song
template during a short critical period in their life suggesting that auditory
encoding of the song template and the process of vocal imitation are two
separate events'' \todo{source?}

\subsection{Day learning model}

* hillclimbing
* Optimize gesture but not song structure


\subsection{Night learning model}

* Microbial GA
* Optimise only song structure, not gestures

\subsection{Satellite Modules}

\subsubsection{Bird song analysis package}

* Imported from matlab

\subsubsection{Synthesizer}

* Cythonized and debugged

\chapter{Results}

\section{Syllables development trajectories}

\section{Influence of song model restructuration on learning}

\chapter{Discussion}

\chapter{Conclusion}

\printbibliography{}

\end{document}
